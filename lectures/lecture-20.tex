\documentclass{article}
\usepackage{graphics}
\usepackage{psfig}
\newcommand{\be}{\begin{equation}}
\newcommand{\ee}{\end{equation}}
\newcommand{\bue}{\begin{equation}}
\newcommand{\eue}{\end{equation}}
\newcommand{\bc}{\begin{center}}
\newcommand{\ec}{\end{center}}
\newcommand{\bea}[1]{\begin{eqnarray}\label{#1}}
\newcommand{\eea}{\end{eqnarray}}
\newcommand{\bua}{\begin{eqnarray*}}
\newcommand{\eua}{\end{eqnarray*}}
\newcommand{\dd}[2]{{{d#1}\over{d#2}}}
\newcommand{\ddt}[1]{\dd{#1}{t}}
\newcommand{\dddt}[1]{\dd{^2#1}{t^2}}
\newcommand{\aver}[1]{\langle{#1}\rangle}
\newcommand{\atom}[3]{\ifmmode^{#1}_{#2}{\rm{#3}}\else{$^{#1}_{#2}${#3}}\fi}
\newcommand{\electron}{\atom{~0}{-1}{e}}
\newcommand{\positron}{\atom{0}{1}{\bar{e}}}
\newcommand{\neutrino}{\atom{0}{0}{\nu}}
\newcommand{\photon}{\atom{0}{0}{\gamma}}
\newcommand{\antineutrino}{\atom{0}{0}{\bar{\nu}}}
\newcommand{\neutron}{\atom{1}{0}{n}}
\newcommand{\proton}{\atom{1}{1}{p}}
\newcommand{\hydrogen}{\atom{1}{1}{H}}
\newcommand{\deuterium}{\atom{2}{1}{H}}
\newcommand{\tritium}{\atom{3}{1}{H}}
\newcommand{\helium}{\atom{4}{2}{He}}
\newcommand{\hethree}{\atom{3}{2}{He}}
\begin{document}
\def\half{{{1}\over{2}}}
\def\third{{{1}\over{3}}}
\def\mv2{{mv^2}}
\def\vece{{\mathbf e}}
\def\vecv{{\mathbf v}}
\def\vecr{{\mathbf r}}
\def\vecp{{\mathbf p}}
\def\vecF{{\mathbf F}}
\def\vecL{{\mathbf L}}
\def\bra{\left[}
\def\ket{\right]}
\def\densu{kg/m$^3$}
\def\ergu{J/m$^3$}
\def\pressu{kg/m$^3$}
\def\velu{km/s}
\def\hubu{km/s/Mpc}
\def\msol{M$_{S}$}
\def\lsol{L$_{S}$}
\def\rsol{R$_{S}$}
\def\mearth{M$_{E}$}
\def\rearth{R$_{E}$}
\subsection*{Lecture notes 20: Inflation}
The observed galaxies, quasars and supernovae, as well as observations of intergalactic absorption lines, tell us about the state of
the universe during the period where $z<6$ when the universe was older than 1~Gyr. The observed CMB tells us the state of the universe
at around $z=1100$ when the Universe was some $t\approx 0.35$~Myr. The observed abundances of the light elements allow us to probe the 
universe at the time of nucleosynthesis; $z=3\times 10^8$, $t\simeq 3$~min. Finally the time of proton freeze-out; $z=4\times 10^9$, 
$t\simeq 1$~s, is evidenced by the helium abundance in the universe of $Y=0.24$ indicating that we understand the universe even at this
early date. 
\par
There are however three problems with the state of the universe today that are not easily reconciled with a theory that {\it necessarily}
leads to the universe we live in today. 
\subsubsection*{Flatness}
The Friedmann equation may be written as an equation for the curvature
\[ 1-\Omega(t)=-{\kappa c^2\over R_0^2a(t)^2H(t)^2} \]
today this equation reads
\[ 1-\Omega_0=-{\kappa c^2\over R_0^2H_0^2}. \]
Dividing one by the other and rearranging gives us an equation for the critical density as
\[ 1-\Omega(t)={H_0^2(1-\Omega_0)\over H(t)^2a(t)^2} \]
In the benchmark model we have $\Omega_{r,0}=8.4\times 10^{-5}$, $\Omega_{m,0}=0.3$ and
$\Omega_\Lambda=0.7$. Thus, observations of SN~Ia and the CMB tell us that $|1-\Omega_0|\leq 0.2$, which implies a universe that is close 
to flat. Let us see how this parameter develops as we move back in time.
\par
Another way of writing the Friedmann equation is
\[ {H^2\over H_0^2}={\epsilon\over\epsilon_0}+{1-\Omega_0\over a^2} \]
where we can expand the $\epsilon/\epsilon_0$ to 
\[ {\epsilon\over\epsilon_0}={\Omega_{r,0}\over a^4}+{\Omega_{m,0}\over a^3}+{\Omega_{\Lambda}}\approx {\Omega_{r,0}\over a^4}+{\Omega_{m,0}\over a^3} \quad {\rm for} \quad a<<1 \]
Which gives
\[ 1-\Omega(t)={(1-\Omega_0)a^2\over\Omega_{r,0}+a\Omega_{m,0}} \]
During the radiation dominated phase we have $|1-\Omega|_r\propto a^2\propto t$ while in the matter dominated phase 
$|1-\Omega|_m\propto a\propto t^{2/3}$. 
\par
Using the number above for the critical density today at $t_0$ we find that the density parameter was 
\[ |1-\Omega_{rm}|\leq 2\times 10^{-4} \]
at the time of radiation-matter equality, 
\[ |1-\Omega_{nuc}|\leq 3\times 10^{-14} \]
at the time deuterium was forming. If we push the extrapolation back to the Planck time $t_p\approx 5\times 10^{-44}$~s we find 
\[ |1-\Omega_{P}|\leq 1\times 10^{-60}. \]
{\it Why in the world is the universe flat to such an incredible high precision?? There is nothing in the laws of physics we know telling 
us it must be so.}
\subsubsection*{The Horizon}
The universe was matter dominated at the time of the last scattering, so the horizon distance can be approximated by 
\[ d_{hor}=2{c\over H(t_{ls})} \]
We have already seen that the Hubble distance at this time was $0.2$~Mpc so the horizon distance was only $d_{hor}\approx 0.4$~Mpc: Points
further apart were not in causal contact if the universe were radiation or matter dominated up to this time. The angular distance to the last
scattering surface is $d_A\approx 13$~Mpc (Why?). Thus points in the CMB further apart than 
\[ \theta_{hor}={d_{hor}\over d_A}\approx 0.03~{\rm rad}\approx 2^o \]
as seen from the Earth today should never have been exchanging information. There are $20\,000$ such a-causal patches on the sky which 
nevertheless have temperatures that do not vary by much more than $1/100\,000$. 
\par
{\it This seems a highly unlikely state of affairs!}
\subsubsection*{Monopoles}
One of the major projects of modern physics is the unification of all four forces into one force. It has already been shown that the 
weak interaction and the electromagnetic force are one and the same at energies above $E_{ew}\simeq 1$~TeV, or equivalently 
$T_{ew}\simeq 10^{16}$~K, which for the Universes part means that at times before an age of $t\simeq 10^{-12}$~s these forces were one
and the same, the {\bf electro-weak} force. Similarly, it is thought that above an even higher energy --- super-symmetry theory, which is 
not well confirmed, indicates an energy for {\bf grand unification} of $E_{GUT}\simeq 10^{12} - 10^{13}$~TeV or temperatures of 
$T_{GUT}\simeq 10^{28}$~TeV, which would indicate an age of less than some $t_{GUT}\simeq 10^{-36}$~s. 
\par
(This is four order of magnitude below the {\bf Planck scale}: $E_P\simeq 10^{16}$~TeV, which arises when one sees at which mass $m_P$ the 
Schwarzschild radius equals the Compton length, {\it i.e.}
\[ R_{Sch}={m_{P}G\over c^2} \qquad \lambda_C={\hbar\over m_Pc} \quad \Rightarrow \quad m_P=\left({\hbar c\over G}\right)^{1/2}=2.2\times 10^{-8}~{\rm kg} \]
The length that this occurs at defines the Planck length
\[ l_P=\left({G\hbar\over c^3}\right)^{1/2}=1.6\times 10^{-35}~{\rm m} \]
while, the time it takes to cross this distance at the speed of light defines the Planck time
\[ t_P=\left({G\hbar\over c^5}\right)^{1/2}=5.4\times 10^{-44}~{\rm s} \]
It is though the an eventual theory of quantum gravity will unify with the other forces at the Planck scale.)
\par
As the universe cools below $T_{GUT}$, a phase transition occurs when the strong force separates from the electro-weak force. Points further
away from each other than the horizon size at that time, $t_{GUT}$, will be out of causal contact and it is expected that {\bf topological
defects} will form between regions the size of the horizon $d_{hor}={2 c/H_{GUT}}=2ct_{GUT}$. According to super-symmetric theory these
defects will take the form of {\bf magnetic monopoles} (or perhaps cosmic strings?). The density of such monopoles should then be
\[ n_M(t_{GUT})\simeq{1/(2ct_{GUT})^3}\simeq 10^{82}~{\rm m^{-3}} \]
It is also predicted that the mass of such monopoles is of order $E_{GUT}/c^2$, thus the energy density of monopoles is predicted to be
\[ \epsilon_{M,GUT}\simeq E_{GUT}n_{M,GUT}\simeq 10^{94}~{\rm TeV/m^3} \]
The photon energy at that time was 
\[ \epsilon_{\gamma,GUT}\simeq \alpha T_{GUT}^4\simeq 10^{104}~{\rm TeV/m^3} \]
but since monopoles are non-relativistic their energy density falls, like other cold matter, as $1/a^3$ instead of the $1/a^4$ that the 
photon/neutrino energy density falls at. Thus, when the temperature falls below $T=10^{-10}T_{GUT}\simeq 10^{18}$~K at an age of
$t\simeq 10^{-16}$~s, we would expect that monopoles dominate the universe. This is certainly not the case today, no one has found a
magnetic monopole.
\par
{\it Where have all the monopoles gone?}
\subsubsection*{The Inflation solution}
When Einstein first solved his field equations for the universe he found that he could not develop a static non-trivial solution without
introducing an extra term into his equations a {\bf cosmological constant} $\Lambda$. This term works as a repulsive force on large spatial
scales and thus keeps the universe from collapsing under its own weight (though it must be admitted that the universe with a cosmological
constant is {\it unstable}; given a small perturbation an initially static universe would collapse or expand). In terms of the Friedmann
equation the cosmological constant appears on the right hand side
\[ \left({\dot{a}\over a}\right)^2={8\pi G\over 3c^2}\epsilon-{\kappa c^2\over R_0^2a^2}+{\Lambda\over 3} \]
We may rewrite this term by introducing the energy density associated with the cosmological constant
\[ \epsilon_\Lambda={c^2\over 8\pi G}\Lambda. \]
Note that if we require the cosmological constant to remain constant ($\dot{\epsilon}=0$), the fluid equation requires that 
$P_\Lambda=-\epsilon_\Lambda$. Note also that the static Einstein Universe ($\dot{a}=0$) is closed with a radius of curvature 
\[ R_0={c\over 2(\pi G\rho)^{1/2}}={c\over\Lambda^{1/2}}. \]
\par
In the mid 1980's Alan Guth came up with the idea that the universe had suffered an {\bf inflationary expansion} early in its history in 
order to explain the problems outlined above. Assume that one has a cosmological constant $\Lambda_i$ sufficient to dominate universal 
dynamics. We can then write the Friedmann equation
\[ \left({\dot{a}\over a}\right)^2={\Lambda_i\over 3}. \]
The Hubble parameter is $H_i=({\Lambda_i/3})$ which has solution
\[ a(t)=e^{H_it} \]
Let us now further suppose an inflationary era from time $t_i$ until time $t_f$, before and after which the universe is radiation dominated. 
The expansion factor is then
\[ a(t) =\left\{ \begin{array}{ll}
             a_i(t/t_i)^{1/2} & \mbox{for $t<t_i$}; \\
             a_ie^{H_i(t-t_i)} & \mbox{for $t_i<t<t_f$}; \\
             a_ie^{H(t_f-t_i)}(t/t_f)^{1/2} & \mbox{for $t>t_f$}.\end{array}\right. \]
The universe has expanded a factor
\[ {a(t_f)\over a(t_i)}=e^{N} \quad {\rm where} \quad N=H_i(t_f-t_i) \]
during inflation. We will in the following assume that $N\simeq 100$.
\par
What happens to the flatness of the universe during inflation? The Friedmann equation says
\[ |1-\Omega|={c^2\over a^2H^2R_0^2}\propto e^{-2H_it} \]
thus, any initial curvature (for example $|1-\Omega|\simeq 1$) is reduced by an enormous factor, of order $10^{-87}$ using the numbers above,
during inflation.
\par 
How about the horizon problem? The size of the horizon is in general given by 
\[ d_{hor}(t)=a(t)c\int_0^t{dt\over a(t)} \] 
which for $t_i$ is $d_{hor}(t_i)=2ct_i$ since the universe is radiation dominated
up till this time. The horizon at $t_f$ is easily calculated using the above:
\[ d_{hor}(t_f)=a_ie^Nc\left(\int_0^{t_i}{dt\over a_i(t/t_i)^{1/2}}+\int_{t_i}^{t_f}{dt\over a_i\exp\bra H_i(t-t_i)\ket}\right) \simeq ce^N(2t_i+H_i^{-1}) \]
Assuming that $t_i=10^{-36}$~s, that $H_i=10^{36}$~s, and $N=100$ we find that the horizon goes from $d_{hor}(t_i)=6\times 10^{-28}$~m to
$d_{hor}(t_f)\simeq e^N3ct_i\simeq 2\times 10^{16}~{\rm m}\simeq 0.8~{\rm pc}$, and that within $10^{-34}$~s!! Everything we see inside
our horizon is causally connected if inflation has happened.
\par
The exponential growth of space also spreads any eventual monopoles so greatly that there are less than one monopole within our horizon today.
\par
Does this early inflation have anything at all to do with the inflation we see today? Probably not: The energy density of today's
cosmological constant is 
\[ \epsilon_{\Lambda,0}\approx 0.7\epsilon_{c,0}\approx 0.04~{\rm TeV/m^3} \]
while in the universe's infancy we find (again using the numbers above\ldots)
\[ \epsilon_{\Lambda_i}={c\over 8\pi G}H_i^2\simeq 10^{105}~{\rm TeV/m^3} \]
\subsection*{Exercises: Inflation}
\vspace {\baselineskip}
\begin{enumerate}
\item What is the upper limit placed on $\Omega(t_P)$ by the requirement that the universe not
end in a Big Crunch between the Planck time, $t_P\approx 5\times 10^{-44}$~s, and the start of 
the inflationary epoch at $t_i$? Compute the maximum permissible value of $\Omega(t_P)$, first
assuming $t_i\approx 10^{-36}$~s, then assuming $t_i\approx 10^{-26}$~s. (Hint: prior to 
inflation, the Friedmann equation will be dominated by the radiation term and the curvature 
term.)
\item Current observational limits on the density of magnetic monopoles tell us that their
density parameter is currently $\Omega_{M,0}<10^{-6}$. If mononpoles formed at the GUT time,
with one momonpole per horizon of mass $m_M=m_{\rm GUT}$, how many e-foldings of inflation
would be required to drive the current of monopoles below the limit $\Omega_{M,0}<10^{-6}$?
Assume that inflation took place immediately after the formation of monopoles.
\item It has been speculated that the present-day acceleration of the universe is due to the
existence of a false vacuum, which will eventually decay. Suppose that the energy density
of the false vacuum is $\epsilon_{\Lambda}=0.7\epsilon_{c,0}=3600$~MeV~m$^{-3}$, and that the
current energy density of matter is $\epsilon_{m,0}=0.3\epsilon_{c,0}=1600$~MeV~m$^{-3}$. What
will be the value of the Hubble parameter once the false vacuum becomes strongly dominant?
Suppose that the false vacuum is fated to decay instantaneously to radiation at a time 
$t_f=50t_0$. (Assume, for simplicity, that the radiation takes the form of blackbody photons.)
To what temperature will the universe be reheated at $t=t_f$? What will the energy density
of matter be at $t=t_f$? At what time will the universe again be dominated by matter?
\end{enumerate}
\end{document}