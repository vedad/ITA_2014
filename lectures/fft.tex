Let us now estimate a Fourier transform form a finit number of sampled points.
Suppose that we have $N$ conscutive sampled values
\[
h_k\equiv h(t_k) \qquad t_k\equiv k\Delta \qquad k=0,1,2,\ldots,N-1 
\]
so that the sampling interval is $\Delta$. Also assume $N$ is even. With $N$ 
numbers of input, we cannot produce more than $N$ independent numbers of 
output. So, instead of trying to estimate the Fourier transform $H(\nu)$ at 
all values of $\nu$ in the range $-\nu_c$ to $\nu_c$, let us seek estimates 
at only the discrete values
\[
\nu_n\equiv{n\over N\Delta},\qquad n={-N\over 2},\ldots,{N\over 2} 
\]
The extreme values of $n$ correspond to the lower and upper values fo the 
Nyquist critical frequency range. 

The remaining step is to approximate the continuous tranform by a discrete 
sum
\[
H(\nu_n)=\infint h(t)e^{2\pi i \nu_nt}dt\approx \sum_{k=0}^{N-1}h_ke^{2\pi i\nu_nt_k}\Delta=\Delta\sum_{k=0}^{N-1}h_k e^{2\pi ikn/N}
\]
The final summation is called the {\it discrete Fourier transform} of the $N$ 
points $h_k$. Let us denote it by $H_n$:
\be
H_n\equiv\sum_{k=0}^{N-1}h_k e^{2\pi ikn/N}
\label{eq:discret-fourier}
\ee
Up to now we have taken the view that index $n$ varies from $-N/2$ to $N/2$. 
However, since it is clear that equation~\ref{eq:discrete-fourier} is 
periodic in $n$ we also have that $H_{-n}=H_{N-1}$. With this in mind it is 
customary to let $n$ vary from $0$ to $N-1$ (one period). When this convention
is followed the {\it zero} frequency corresponds to $n=0$, positive 
frequencies $0<\nu<\nu_c$ correspond to valuses $1\le n\le {N/2}-1$, while
negative frequencies $-\nu_c<\nu<0$ correspond to ${N/2}_1\le n \le N-1$. The
value $n={N/2}$ corresponds to both $\nu=\nu_c$ and $\nu=-\nu_c$.

The discrete {\it inverse} transform which recovers $h_k$ form the $H_n$ is
\[
h_k={1\over N}\sum_{n=0}^{N-1}H_n e^{-2\pi ikn/N}
\]

How do we implement the discrete transform in code? The brute strenght 
approach takes of order $N^2$ operations: Define $W$ as the complex nubmer
\[
W\equiv e^{2\pi i/N}
\]
Then equation~\ref{eq:discrete-fourier} can be written 
\[
H_n=\sum_{k=0}^{N-1}W^{nk}h_k,
\]
{\it i.e.} the vector $h_k$ is multiplied by a matrix whose $(h,k)^{th}$
element is the constant $W$ to the power $n\times k$. The matrix multiplication
produces a vector result whose components are $H_n$. This multiplication
evidently needs $N^2$ complex multiplications. 

In fact, the discrete Fourier transform can be achieved in $N\log_2 N$ 
operations with an algorithm called the {it Fast Fourier Transform} or 
{\it FFT}. Here is one derivation of the FFT, that of Danielson and Lanczos
in 1942. They showed that a discrete transform of length $N$ can be rewritten
as the sum of two discrete transforms, each of length $N/2$. One of the 
two is formed from the even numbered points of the original $N$, the other 
from the odd-numbered points.
\bau
F_k & = & \sum_{j=0}^{N-1}e^{2\pi ijk/N}f_j \\
    & = & \sum_{j=0}^{{N/2}-1}e^{2\pi i2jk/N}f_{2j}
         +\sum_{j=0}^{{N/2}-1}e^{2\pi i(2j+1)k/N}f_{2j+1} \\
    & = & \sum_{j=0}^{{N/2}-1}e^{2\pi ijk/({N/2})}f_{2j}
         +W^k\sum_{j=0}^{{N/2}-1}e^{2\pi ijk/({N/2})}f_{2j+1} \\
    & = & F_k^e+W^kF_k^o
\eau
$F_k^e$ denotes the $k^{th}$ component of the Fourier transform of length 
$({N/2})$ formed from the even components of the original $f_j$, while $F_k^o$
is the corresponding transform of lenght $({N/2})$ from the odd components.

This procedure can be applied recursively; having reduced the problem of
computing $F_k$ to that of computing $F_k^e$ and $F_k^o$, one can do the 
same reduction of $F_k^e$ to the problem of computing the transform of 
{\it its} $N/4$ even-numbered inputs data, $F_k^{ee}$, and $N/4$ odd-numbered 
data $F_k^{eo}$. When we start with an original $N$ which is a integer 
power of 2 (one should only use FFTs with $N$ a power of 2, padding the input
data with zeroes is necessary) we can continue applying the Danielson-Lanczos 
method until the original data is subdivided all the way down to transforms
of length one. The Fourier transfrom of lenght one is just the identity 
operation that copies its one input number into its output slot. Thus, 
for every pattern of $e$'s and $o$'s numbering $\log_2 N$ in all there is a 
one-point transform that is given by 
\[
F_k^{eoeeoeo\cdots oee}=f_n\qquad {\rm for~some~}n
\]
Which value of $n$ corresponds to which pattern? Reverse the pattern of 
$e$'s and $o$'s, let $e=0$ and $o=1$ and one will have, {\it in binary}, the
value of $n$. This is because the successive subdivisions of the data into 
odd and even are tests of successive least significant bits of $n$. Thus
by rearranging the input vector $f_j$ in bit reversed order so that the 
indivdual numbers are in the order obtained by bit reversing $j$. The points
are given as one-point transforms. These are recombined with the adjacent 
number to give two-point transforms, then combine adjacent pairs again to 
give 4-point transforms, and so on, using the Danielson-Lanczos formula
at every step.

The FFT therefore has a structure where first the data are sorted into
bit reversed order and thereafter the transforms of length $2,4,8,/ldots,N$
are computed implementing the Danielson-Lanczos formula.

